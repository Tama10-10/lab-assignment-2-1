\documentclass{article}

% Language setting
% Replace `english' with e.g. `spanish' to change the document language
\usepackage[english]{babel}

% Set page size and margins
% Replace `letterpaper' with `a4paper' for UK/EU standard size
\usepackage[letterpaper,top=2cm,bottom=2cm,left=3cm,right=3cm,marginparwidth=1.75cm]{geometry}

% Useful packages
\usepackage{amsmath}
\usepackage{graphicx}
\usepackage[colorlinks=true, allcolors=blue]{hyperref}

\title{Data Structure: Theoretical Approach}
\author{Durgesh Raghuvanshi}

\begin{document}
\maketitle
\title{B-Tech Department of Computer Science,
IILM Academy of Higher Learning, Greater Noida, Uttar Pradesh, India}\\
\begin{}
ABSTRACT\\
Run with accordance with significance. The first if
these this paper explains about the basic terminologies
used in this paper in data structure. Better running
times will be other constraints, such as memory use
which will be paramount. The most appropriate data
structures and algorithms rather than through hacking
removing a few statements by some clever coding.
Data structures serve as the basis for abstract data
types (ADT). "The ADT defines the logical form of
the data type. The data structure implements the
physical form of the data type."Different types of data
structures are suited to different kinds of applications,
and some are highly specialized to specific tasks. For
example, relational databases commonly use B-tree
indexes for data retrieval, while compiler
implementations usually use hash tables to look up
identifiers.\\
\end{abstract}

\section{Introduction}

Data structures serve as the basis for abstract data
types (ADT). "The ADT defines the logical form of
the data type. The data structure implements the
physical form of the data type."Different types of data
structures are suited to different kinds of applications,
and some are highly specialized to specific tasks. For
example, relational databases commonly use B-tree
indexes for data retrieval, while compiler
implementations usually use hash tables to look up
identifiers. Data structures provide a means to manage
large amounts of data efficiently for uses such as large
databases and internet indexing services. Usually,
efficient data structures are key to designing efficient
algorithms. Some formal design methods and
programming languages emphasize data structures,
rather than algorithms, as the key organizing factor in
software design. Data structures can be used to
organize the storage and retrieval of information
stored in both main memory and secondary memory.
Data structures are generally based on the ability of a
computer to fetch and store data at any place in its
memory, specified by a pointer—a bit string,
representing a memory address, that can be itself
stored in memory and manipulated by the program.
Thus, the array and record data structures are based on
computing the addresses of data items with arithmetic
operations, while the linked data structures are based
on storing addresses of data items within the structure
itself. Many data structures use both principles,
sometimes combined in non-trivial ways (as in XOR
linking).[citation needed]
The implementation of a data structure usually
requires writing a set of procedures that create and
manipulate instances of that structure. The efficiency
of a data structure cannot be analyzed separately from
those operations. This observation motivates the
theoretical concept of an abstract data type, a data
structure that is defined indirectly by the operations
that may be performed on it, and the mathematical
properties of those operations (including their space
and time cost).[citation needed]An array is a number
of elements in a specific order, typically all of the
same type (depending on the language, individual
elements may either all be forced to be the same type,
or may be of almost any type). Elements are accessed
using an integer index to specify which element is
required. Typical implementations allocate contiguous
memory words for the elements of arrays (but this is
not necessity). Arrays may be fixed-length or
resizable. A linked list (also just called list) is a linear
collection of data elements of any type, called nodes,
where each node has itself a value, and points to the
next node in the linked list. The principal advantage
of a linked list over an array, is that values can always
be efficiently inserted and removed without relocating
the rest of the list. Certain other operations, such as
random access to a certain element, are however
slower on lists than on arrays. Most assembly languages and some low-level languages, such as
BCPL (Basic Combined Programming Language),
lack built-in support for data structures. On the other
hand, many high-level programming languages and
some higher-level assembly languages, such as
MASM, have special syntax or other built-in support
for certain data structures, such as records and arrays.

\section{Sequential search}

When data items are stored in a collection such as a
list, we say that they have a linear or sequential
relationship. Each data item is stored in a position
relative to the others. In Python lists, these relative
positions are the index values of the individual items.
Since these index values are ordered, it is possible for
us to visit them in sequence. This process gives rise to
our first searching technique, the sequential search.
Starting at the first item in the list, we simply move
from item to item, following the underlying sequential
ordering until we either find what we are looking for
or run out of items. If we run out of items, we have
discovered that the item we were searching for was
not present.
\begin{table}[ht]
    \centering
    \begin{tabular}{|c|c|c|}
    \hline
        Algorithm & Best  case  & Expected\\
       \hline
       Selection sort & O(N2) & O(N2)\\
       \hline
           Merge sort &O(NlogN) &O(NlogN)\\
       \hline
         Linear search& O(1) & O(N)\\
         \hline
         Binary search& O(1) & O(logN)\\
         \hline
    \end{tabular}
    \caption{Sequential search}
    \label{tab:my_label}
\end{table}
\section{Depth of node}
The depth of node is the length of the path from the
root to the node. A rooted tree with only one node has
a depth of zero.


\section{Threaded binary tree}

In a threaded binary tree all the null pinters which
wasted the space in linked representation is converted
into useful links called threads thus representation of a
binary tree using these threads is called threaded
binary tree. 
\section{Analysis of sequential search}

To analyze searching algorithms, we need to decide
on a basic unit of computation. Recall that this is
typically the common step that must be repeated in
order to solve the problem. For searching, it makes
sense to count the number of comparisons performed.
Each comparison may or may not discover the item
we are looking for. In addition, we make another
assumption here. The list of items is not ordered in
any way. The items have been placed randomly into
the list. In other words, the probability that the item
we are looking for is in any particular position is
exactly the same for each position of the list.
If the item is not in the list, the only way to know it is
to compare it against every item present. If there are
\(n\) items, then the sequential search requires \(n\)
comparisons to discover that the item is not there. In
the case where the item is in the list, the analysis is
not so straightforward. There are actually three
different scenarios that can occur. In the best case we
will find the item in the first place we look, at the
beginning of the list. We will need only one
comparison. In the worst case, we will not discover
the item until the very last comparison, the nth
comparison.

\section{Binary search}

Binary search is a fast search algorithm with run-time
complexity of Ο(log n). This search algorithm works
on the principle of divide and conquer. For this
algorithm to work properly, the data collection should
be in the sorted form.
Binary search looks for a particular item by
comparing the middle most item of the collection. If a
match occurs, then the index of item is returned. If the
middle item is greater than the item, then the item is
searched in the sub-array to the left of the middle
item. Otherwise, the item is searched for in the subarray to the right of the middle item. This process
continues on the sub-array as well until the size of the
sub array reduces to zero. B-trees are generalizations
of binary search trees in that they can have a variable
number of sub trees at each node. While child-nodes
have a pre-defined range, they will not necessarily be
filled with data, meaning B-trees can potentially waste
some space. The advantage is that B-trees do not need
to be re-balanced as frequently as other self-balancing
trees.
Due to the variable range of their node length, B-trees
are optimized for systems that read large blocks of
data. They are also commonly used in databases. A
ternary search tree is a type of tree that can have 3
nodes: a lo kid, an equal kid, and a hi kid. Each node
stores a single character and the tree itself is ordered
the same way a binary search tree is, with the
exception of a possible third node. Searching a ternary
search tree involves passing in a string to test whether
any path contains it. The time complexity for
searching a balanced ternary search tree is O(log n).
\begin{figure}
    \centering
    \includegraphics{Bee.png}
    \caption{3D view of Bee}
    \label{fig:my_label}
\end{figure}
\section{Here is equation}
Equation is $(a+b)^2=a^2+2ab+b^2$
\section{conclusion}
This paper covered the basics of data structures. With
this we have only scratched the surface.
Although we have built a good foundation to move
ahead. Data Structures is not just limited to Stack,
Queues, and Linked Lists but is quite a vast area.
There are many more data structures which include
Maps, Hash Tables, Graphs, Trees, etc. Each data
structure has its own advantages and disadvantages
and must be used according to the needs of the
application. A computer science student at least know
the basic data structures along with the operations
associated with them. Many high level and object
oriented programming languages like C#, Java,
Python come built in with many of these data
structures. Therefore, it is important to know how
things work under the hood. Dynamic data structures
require dynamic storage allocation and reclamation.
This may be accomplished by the programmer or may
be done implicitly by a high-level language. It is
important to understand the fundamentals of storage
management because these techniques have
significant impact on the behavior of programs. The
basic idea is to keep a pool of memory elements that
may be used to store components of dynamic data
structures when needed. Allocated storage may be
returned to the pool when no longer needed. In this
way, it may be used and reused. This contrasts sharply
with static allocation, in which storage is dedicated for the use of static data structures. It cannot then be
reclaimed for other uses, even when no needed for the
static data structure. As a result, dynamic allocation
makes it possible to solve larger problems that might
otherwise be storage-limited. Garbage collection and
reference counters are two basic techniques for
implementing storage management. Combinations of
these techniques may also be designed. Explicit
programmer control is also possible. Potential pitfalls
of these techniques are garbage generation, dangling
references, and fragmentation. High-level language
may take most of the burden for storage management
from the programmer. The concept of pointers or
pointer variables underlies the use of these facilities,
and complex algorithms are required for their
implementation.
\bibliographystyle{alpha}
\bibliography{sample}
1. Book of Data structures through C G. S Baluja.\\
2. Pieren Garry Department of computer science
New York University.\\
3. Paul Xavier department of algorithms in c
Amsterdam.\\
4. Surendrakumar Ahuja IItdelhi department of
computer science delhi .\\
5. Nick jones department of data mining Australia.
\end{document}\\